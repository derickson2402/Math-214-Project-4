\documentclass[12pt]{article}
%\usepackage{anysize}
%\usepackage{dsfont}
\usepackage{fullpage}
\usepackage{verbatim}
\usepackage{amsmath}
\usepackage{hyperref}
\usepackage{url,amsfonts, amssymb, amsthm,color, enumerate, multicol, tikz}
\usepackage{placeins}
\usepackage{listings}
\usepackage{textcomp}
\usepackage{multicol}
\usepackage{bookmark}
% \usepackage{logicproof}
\usepackage{xcolor}
\usepackage{colortbl}
\usepackage{scrextend}
\newcommand{\filcl}{\cellcolor{gray!25}}
%\marginsize{2cm}{2cm}{2cm}{3cm}
\renewcommand{\iff}{\leftrightarrow}

% Configure graphicx package
\usepackage{graphicx, blindtext, float}
\graphicspath{ {./images/} }
\setlength{\parskip}{0pt}

\newenvironment{answer} {
	\setlength{\parindent}{0pt}
	\setlength{\parskip}{6pt}
	\vspace{6pt}
	\begin{addmargin}{1cm}
} {
	\end{addmargin}
	\vspace{12pt}
	\setlength{\parindent}{12pt}
}

\title{ Math 214 Linear Algebra, Project 4 Capstone }
\date{\today}
\author{Dan Erickson, Jessica Zhu, Ibrahim Alnassar, Tim Sawyer}

\begin{document}
\maketitle

These days, it is commonplace to share hundreds if not thousands of pictures over the internet each day, whether through social media, Zoom calls, or sharing spring break pictures with family. Storing and sending this many images takes a lot of storage and internet bandwidth, it is hugely important to reduce the original size of digital images in a way that retains as much detail as possible. We will explore the application of Linear Algebra in digital image compression, specifically the methods using wavelet-based algorithms. Our goal is to look at how the JPEG-2000 compression standard is implemented, why it performs better than older standards especially at higher compression ratios, and how the process could potentially be improved.

\section{Introduction of Topic}

For the past 30 years, The JPEG image compression standard has been the gold standard for digital images. Eight years after its release, the JPEG-2000 standard was released. It uses wavelet transformation compression rather than discrete cosine transformation, which allows it to achieve much more efficient compression without losing as much perceivable detail. This means that end users can store and transport more images, while keeping them at a higher quality.

Wavelet image compression utilizes a type of Fourier transformation to reduce the image to a series of basis functions. The original image can be created by simply summing all of the functions together. This process is essentially a two-dimensional version of the Fourier transformations often used in audio signal spectrograms. With the image collapsed into a series of functions, any functions deemed low cost functions (as determined by the compression level and the image itself) can be thrown out in order to reduce the total amount of data without greatly affecting the original image. The image is then reconstructed, and even compression levels of $ 90\% $ or higher will not alter the image enough for a human to perceive the difference.

There are dozens of whitepapers and technical documents about wavelet compression, not to mention the original JPEG-2000 specification. However, these are not very interesting, so we will rely on a video series on YouTube by Steve Brunton from the University of Washington for learning about this topic. The video series can be found at this link \url{https://youtube.com/playlist?list=PLMrJAkhIeNNT_Xh3Oy0Y4LTj0Oxo8GqsC} and the relevant videos start at "Wavelets and Multiresolution Analysis".

\section{Data Collection}

% TODO: What data will you use?

\section{Project Management}

Our group consists of the following members:

\begin{list}{}{}
	\item Dan Erickson - danerick - Computer Science Engineering
	\item Jessica Zhu - jesszhu - Computer Science Engineering
	\item Ibrahim Alnassar - alnassar - Computer Science Engineering
	\item Tim Sawyer - tisawyer - Industrial and Operations Engineering
	\item Josh Richman - richmajo - Data Science Engineering
\end{list}

We have a very strong programming, mathematics, and data analysis background.

% TODO: What background and skills will each team member bring to the project?

We all have very busy lives and will have other final projects and exams due around the same time as this project. Therefore, we will aim to complete our poster a week early. We will probably not completely finish by then, but we will have the majority done so we can focus on exams.

\end{document}
